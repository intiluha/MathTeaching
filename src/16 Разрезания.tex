\documentclass[a4paper,12pt]{article}
\usepackage{school}
\usetikzlibrary {backgrounds}

\begin{document}
    \head{20 февраля}{Разрезания}
    \newrules
    
    \section{Задачи для самостоятельного решения}
    \problem Как от куска материи в $\dfrac{2}{3}$ метра отрезать полметра, не используя измерительных приборов? \\
    \illustrated{[scale=1.1]
        \draw (0, 0) circle [radius=2];
        \draw[fill=black] (2, 0) -- (-2, 0) arc [start angle=180, end angle=360, radius=2];
        \fill[black] (-1, 0) circle [radius=1];
        \fill[white] (1, 0) circle [radius=1];
    }{
        \problem Разделите «инь и янь» одним криволинейным разрезом на две равные фигуры так, чтобы и белые, и чёрные части этих фигур были равны (не просто равновелики).
    } \par \vspace{.5cm}
    \illustrated{[scale=.4]
        \draw[help lines] (0,0) grid (12,9);
        \draw[help lines, fill=white] (2,4) rectangle (10,5);
    }{
        \problem Изображённую на рисунке фигуру («прямоугольник $12 \times 9$ с дыркой $8 \times 1$ в центре») разрежьте на две равные части, из которых можно сложить квадрат.
    }
    \problem \sub Можно ли из листа картона размером $14 \times 14$ вырезать пять квадратов размером $5 \times 5$? \sub А если резать только по линиям сетки? \\
    \illustrated{[scale=5]
        \coordinate (A) at (0,0);
        \coordinate (B) at (60:1);
        \coordinate (C) at (0:1);
        \coordinate (D') at ($(A)!(B)!-10:(C)$);
        \path[name path=AC] (A) -- (C);
        \path[name path=BD'] (B) -- (D');
        \path[name path=AE] (A) -- (40:1);
        \path[name intersections={of=AC and BD', by=D}];
        \path[name intersections={of=AE and BD', by=E}];
        \path[name path=DF] (D) -- +(50:1);
        \path[name path=CF] (C) -- +(130:1);
        \path[name intersections={of=DF and CF, by=F}];
        
        \draw (A) -- (B) -- (C) -- (D) 
        \foreach \p/\q/\n in {A/D/1,D/E/1,A/E/2,B/E/2,D/F/3,B/F/3,C/F/3}{
            \ticks{\p}{\q}{\n}{.5mm}
        };
    }{
        \problem Равносторонний треугольник разрезан на пять равнобедренных треугольников так, как показано на рисунке. Найдите углы при основаниях этих равнобедренных треугольников.
    }
    \problem \sub На сколько треугольников можно разрезать выпуклый пятнадцатиугольник, пользуясь только прямолинейными разрезами? \\
    \sub Покажите, что ответ не изменится, если разрешить разрезать любыми ломаными. \\
    \textit{Возможно, вы захотите сразу решать общий случай. Как вам удобно.} \\
    \illustrated{[scale=1.5]
        \coordinate (A) at (0,0);
        \coordinate (B) at (4,0);
        \coordinate (C) at (2.5,2.8);
        \coordinate (D) at (1,2.5);
        
        \draw[red] ($(A)!.5!(B)$) -- ($(C)!.5!(D)$) ($(B)!.5!(C)$) -- ($(D)!.5!(A)$);
        \draw \foreach \p/\q [count=\n] in {A/B,B/C,C/D,D/A}{
            \ticks{\p}{$(\p)!.5!(\q)$}{\n}{1.5mm}
            \ticks{$(\p)!.5!(\q)$}{\q}{\n}{1.5mm}
        };
    }{
        \problem Выпуклый четырёхугольник разрезали на четыре части по отрезкам, соединяющим середины его противоположных сторон. Докажите, что из этих частей можно составить параллелограмм.
    }
    
    \problem Разрежьте правильный шестиугольник \sub на три равные части; \sub на четыре равные части; \sub на шесть равных частей; \sub на восемь равных частей. \\
    \illustrated{[scale=1.5]
        \coordinate (A) at (.5,0);
        \coordinate (B) at (.5,3);
        \coordinate (C) at (2,2);
        \coordinate (A') at (-.5,0);
        \coordinate (B') at (-.5,3);
        \coordinate (C') at (-2,2);
        
        \draw (A) -- (B) -- (C) -- cycle;
        \draw (0,3.5) -- node[anchor=east] {$l$} (0,-.5);
        \draw (A') -- (B') -- (C') -- cycle;
    }{
        \problem На рисунке показаны равные симметричные относительно прямой $l$ треугольники. Двумя прямолинейными разрезами разделите левый из них на три части, из которых можно сложить правый, если части разрешается только сдвигать и поворачивать, но не переворачивать.
    }
    \problem Докажите, что любой треугольник можно разрезать не более чем на три части, из которых складывается равнобедренный треугольник. \\
    \illustrated{[scale=.75]
        \coordinate (A) at (0.5,0);
        \coordinate (B) at (4.5,0);
        \coordinate (C) at (3,2.8);
        \coordinate (D) at (1.5,2.5);
        \coordinate (A') at (-5+0.5,0);
        \coordinate (B') at (-5+4.5,0);
        \coordinate (C') at (-5+3,2.8);
        \coordinate (D') at (-5+1.5,2.5);
        
        \draw (C) -- (A) -- (B) -- (C) -- (D) -- (A);
        \draw (D') -- (B') -- (C') -- (D') -- (A') -- (B');
    }{
        \problem \sub Два одинаковых выпуклых четырёхугольника разрезаны: первый --- по одной диагонали, а второй --- по другой диагонали. Докажите, что из полученных четырёх треугольников можно сложить параллелограмм.
    } \\
    \illustrated{
        \coordinate (A) at (0,.5);
        \coordinate (B) at (3,.5);
        \coordinate (C) at (2,2);
        
        \draw (2.5,1.25) -- (0,.5) -- (3,.5) -- (2,2) -- (0,.5);
        \draw[xshift=-3.5cm] (1,1.25) -- (3,.5) -- (2,2) -- (0,.5) -- (3,.5);
        \draw[xshift=-7cm] (1.5,.5) -- (2,2) -- (0,.5) -- (3,.5) -- (2,2);
    }{
        \sub Три одинаковых треугольника разрезаны по разным медианам. Сложите из шести полученных треугольников один треугольник.
    }
    \problem Одним прямолинейным разрезом отрежьте от треугольника трапецию, у которой меньшее основание было бы равно сумме боковых сторон.
\end{document}