\documentclass[a4paper,12pt]{article}

\usepackage{school}

\begin{document}
    \head{30 ноября}{Абстрактные группы}
    
    \section{Разбор}
    
    \defn Пусть $X$ --- множество. \emph{Бинарной операцией} на множестве $X$ называется любое отображение $\bullet \colon X \times X \to X$. Хотя по смыслу $\bullet$ это просто функция двух аргументов и её следовало бы записывать $\bullet(g_1, g_2)$, для удобства пишут $g_1 \bullet g_2$, как с арифметическими операциями.
    \example $+$ (сложение) и $\cdot$ (умножение) --- бинарные операции на множестве целых чисел $\Z$.
    \example Пусть $M$ --- любое множество. Тогда $\circ$ (композиция) --- бинарная операция на множестве отображений из $M$ в $M$.

    \defn Непустое множество $G$, снабжённое бинарной операцией $\bullet$, называется \emph{группой}, если выполнены \textbf{аксиомы} \ref{G1}, \ref{G2} и \ref{G3}.
    \begin{description}
        \item[\namedlabel{G1}{(G1)}] Операция $\bullet$ \emph{ассоциативна}:
        \begin{equation*}
            (\forall g, h, k \in G): \quad (g \bullet h) \bullet k = g \bullet (h \bullet k).
        \end{equation*}
        \item[\namedlabel{G2}{(G2)}] Существует \emph{нейтральный} элемент $e_G$ для $\bullet$:
        \begin{equation*}
            (\exists e_G \in G) (\forall g \in G): \quad e_G \bullet g = g = g \bullet e_G.
        \end{equation*}
        \item[\namedlabel{G3}{(G3)}] Каждый элемент в $G$ имеет \emph{обратный} относительно $\bullet$:
        \begin{equation*}
            (\forall g \in G) (\exists g^{-1} \in G): \quad g^{-1} \bullet g = e_G = g \bullet g^{-1}.
        \end{equation*}
    \end{description}
    \textit{Для краткоcти выкладок вы можете опускать значок операции, когда это не приводит к путанице, как с умножением: вместо $(g \bullet h) \bullet k$ писать $(gh)k$. Также можно писать просто $e$ вместо $e_G$.}
    \example Пусть $G = \{e_G\}$ (множество из одного элемента), а $\bullet$ определена единственным возможным образом (каким?). Тогда для $G$ тривиально выполнены все аксиомы, так что $G$ --- группа.
    \problem \sub Докажите, что $\Z$ с операцией $+$ является группой. \\
    \sub Докажите, что $\Z$ с операцией $\cdot$ не является группой.
\begin{solution}
    \sub Ассоциативность сложения целых чисел всем известна. Нейтральный элемент по сложению --- 0. Обратный элемент по сложению к $n$ --- $-n$. \\
    \sub Не существует такого целого числа $2^{-1}$, что $2 \cdot 2^{-1} = 1$ (потому что левая часть равенства всегда делится на $2$, а правая --- нет), так что $(\Z, \cdot)$ --- не группа.
\end{solution}
    \problem \sub Докажите, что в группе только один нейтральный элемент. \\
    \sub Докажите, что у каждого элемента группы только один обратный элемент.
    \begin{solution}
        \sub Пусть $e$ и $e'$ --- нейтральные элементы. Тогда
        \begin{equation*}
            e = e \bullet e' = e',
        \end{equation*}
        где первое равенство следует из нейтральности $e'$, а второе --- из нейтральности $e$. \\
        \sub Пусть $h$ и $k$ --- обратные к $g$. Тогда
        \begin{eqnarray*}
            h  =& h \bullet e \ &\text{(определение нейтрального)} \\
            =& h \bullet (g \bullet k) \ &\text{(определение обратного)} \\
            =& (h \bullet g) \bullet k \ &\text{(ассоциативность)} \\
            =& e \bullet k \ &\text{(определение обратного)} \\
            =& k \ &\text{(определение нейтрального)}.
        \end{eqnarray*}
    \end{solution}
    \problem Докажите, что $D_n = \Sym(P_n)$ (группа симметрий правильного $n$-угольника) в самом деле является группой.
    \begin{solution}
        Знаем, что композиция отображений ассоциативна, а симметрии --- частный случай отображениий, так что \ref{G1} выполнена. Тождественное движение удовлетворяет аксиоме \ref{G2}. В листке про группы симметрий доказывали, что у движений есть обратные, так что \ref{G3} выполнена.
    \end{solution}
    
    \defn Пусть $(G, \bullet)$ --- группа. Подмножество $H \subset G$ называется \mbox{\emph{подгруппой}}, если
    \begin{equation*}
        e_G \in H,
    \end{equation*}
    \begin{equation*}
        (\forall h \in H): \quad h^{-1} \in H,
    \end{equation*}
    \begin{equation*}
        (\forall h_1, h_2 \in H): \quad h_1 \bullet h_2 \in H,
    \end{equation*}
    то есть <<операция не выводит за пределы $H$>>.
    \example В предыдущем листке доказывалось, что произведение чётных перестановок --- чётная перестановка. Следовательно, подмножество всех чётных перестановок из $S_n$ --- подгруппа. Она обозначается $A_n$.
    \problem Найдите все подгруппы в $D_3$, выпишите количества элементов в них.
    \begin{solution}
        Обозначим один из поворотов через $r$, а одну из осевых симметрий через $s$. Тогда $r^2$ --- второй поворот, а $rs$ и $r^2s$ --- вторая и третья симметрии. Таким образом наша группа состоит из следующих элементов: $D_3 = \{e, r, r^2, s, rs, r^2s\}$. \par
        Хотим перечислить всевозможные $H \subset D_3$. Переберём случаи.
        \begin{itemize}
            \item Допустим, $r \in H$. Тогда $r^2 \in H$.
            \begin{itemize}
                \item Допустим, $s \in H$. Тогда $rs, r^2s \in H$, так что $H = D_3$.
                \item Допустим, $s \notin H$. Тогда $rs, r^2s \notin H$, так что $H = \{e, r, r^2\}$.
            \end{itemize}
            \item Допустим, $r \notin H$. Тогда $r^2 \notin H$.
            \begin{itemize}
                \item Допустим, одна из осевых симметрий лежит в $H$. Тогда остальные две не лежат, иначе $r$ и $r^2$ тоже попали бы в $H$. Следовательно $H$ состоит только из этой осевой симметрии и $e$, то есть $H = \{e, s\}$, или $H = \{e, rs\}$, или $H = \{e, r^2s\}$.
                \item Допустим, ни одна из симметрий не лежит в $H$. Тогда $H = \{e\}$.
            \end{itemize}
        \end{itemize}
        Таким образом, все возможные подгруппы: $\{e\}$, $\{e, s\}$, $\{e, rs\}$, $\{e, r^2s\}$, $\{e, r, r^2\}$ и $D_3$.
    \end{solution}
    \problem Пусть $g_1, g_2, g_3, g_4 \in G$, где $(G, \bullet)$ --- какая-то группа. Придайте однозначный смысл выражению $g_1 \bullet g_2 \bullet g_3 \bullet g_4$.
    \begin{solution}
        Значение этого выражения можно определить любым из пяти способов: $((g_1 \bullet g_2) \bullet g_3) \bullet g_4$, $(g_1 \bullet (g_2 \bullet g_3)) \bullet g_4$, $(g_1 \bullet g_2) \bullet (g_3 \bullet g_4)$, $g_1 \bullet ((g_2 \bullet g_3) \bullet g_4)$ или $g_1 \bullet (g_2 \bullet (g_3 \bullet g_4)$. Из-за ассоциативности, ответ получится одним и тем же.
        \begin{eqnarray*}
            (g_1 \bullet (g_2 \bullet g_3)) \bullet g_4 =& ((g_1 \bullet g_2) \bullet g_3) \bullet g_4 \ &\text{(ассоциативность} \\
             =& (g_1 \bullet g_2) \bullet (g_3 \bullet g_4) \ &\text{(ассоциативность} \\
            =& g_1 \bullet (g_2 \bullet (g_3 \bullet g_4)) \ &\text{(ассоциативность)} \\
            =& g_1 \bullet ((g_2 \bullet g_3) \bullet g_4) \ &\text{(ассоциативность)}
        \end{eqnarray*}
    \end{solution}
    \defn Для элемента $g$ группы $(G, \bullet)$ и натурального числа $n$ запись $g^n$ обозначает $\underbrace{g \bullet g \bullet \cdots \bullet g}_{n \text{ раз}}$, а запись $g^{-n}$ обозначает $\underbrace{g^{-1} \bullet g^{-1} \bullet \cdots \bullet g^{-1}}_{n \text{ раз}}$.
    
\begin{theorem} \label{right_cancellation}
    В группах можно сокращать множители справа, то есть для любых трёх элементов $g, h, k$ группы $(G, \bullet)$ имеем
    \begin{equation*}
        g \bullet k = h \bullet k \ \Rightarrow \ g = h.
    \end{equation*}
\end{theorem}
\begin{proof}
    Так как $G$ --- группа, у элемента $k$ существует обратный $k^{-1}$. Тогда имеем
    \begin{eqnarray*}
        g \bullet k =& h \bullet k \ &\Rightarrow \\ 
        (g \bullet k) \bullet k^{-1} =& (h \bullet k) \bullet k^{-1} \ &\Rightarrow \text{(пользуемся ассоциативностью)} \\
        g \bullet (k \bullet k^{-1}) =& h \bullet (k \bullet k^{-1}) \ &\Rightarrow \text{(определение обратного)} \\
        g \bullet e_G =& h \bullet e_G \ &\Rightarrow \text{(определение нейтрального)} \\
        g =& h.
    \end{eqnarray*}
\end{proof}
    \begin{theorem} \label{left_cancellation}
        В группах можно сокращать множители слева, то есть для любых трёх элементов $g, h, k$ группы $(G, \bullet)$ имеем
        \begin{equation*}
            k \bullet g = k \bullet h \ \Rightarrow \ g = h.
        \end{equation*}
    \end{theorem}
    Это первый (но далеко не последний) пример пользы от изучения групп в целом, вместо изучения конкретных интересующих нас групп. Раньше мы отдельно доказывали, что можно сокращать множители в группе симметрий фигуры и в группе перестановок. Теперь мы доказали это разом для всех групп. Так что, когда в будущем нам потребуется изучить какую-нибудь новую группу (например, группу всех вращений куба или группу всех преобразований кубика Рубика), мы сразу будем знать про неё кучу полезных фактов.

    \section{Задачи для самостоятельного решения}
    
    \problem Придумайте ещё 2 бинарных операции на множестве $\Z$.
    \problem Сколько всего бинарных операций на множестве из $n$ элементов?
    
    \problem Докажите, что $S_n$ (группа перестановок $n$ элементов) в самом деле является группой.
    \problem Докажите, что для любых двух элементов $g, h$ любой группы $(G, \bullet)$ выполнено
    \begin{equation*}
        (g \bullet h)^{-1} = h^{-1} \bullet g^{-1}.
    \end{equation*}
    
    \defn Группа $(G, \bullet)$ называется \emph{коммутативной}, если выполнена аксиома \ref{G4}.
    \begin{description}
        \item[\namedlabel{G4}{(G4)}] Операция $\bullet$ \emph{коммутативна}:
        \begin{equation*}
            (\forall g, h \in G): \quad g \bullet h = h \bullet g.
        \end{equation*}
    \end{description}
    \problem \sub Докажите, что группа $A_3$ коммутативна. \\
    \sub Докажите, что группа $A_4$ не коммутативна.
    \Problem{3} Пусть группа $(G, \bullet)$ такова, что для всех $g \in G$ выполнено $g^2 = e_G$. Покажите, что $G$ коммутативна.
    
    \problem \sub Найдите все подгруппы в $S_3$, количества элементов в них.
    \sub Найдите все подгруппы в $A_4$, количества элементов в них.
    
    \problem Докажите теорему \ref{left_cancellation}. \par
    В предыдущих листках мы уже выписывали таблицы умножения для некоторых групп. Можно сделать это для любой группы.
    \begin{center}
        \begin{equation*}
            \begin{array}{c||c|c|c|c}
                \bullet & e & \cdots & h & \cdots \\ \hline \hline
                e & e & \cdots & h & \cdots \\ \hline
                \cdots & \cdots & \cdots & \cdots & \cdots \\ \hline
                g & g & \cdots & g \bullet h & \cdots \\ \hline
                \cdots & \cdots & \cdots & \cdots & \cdots
            \end{array}
        \end{equation*}
    \end{center}
    \problem Покажите, что в каждая строка и каждый столбец содержат все элементы группы ровно по одному разу. \textit{(прямо как судоку!)}
    \problem \sub Покажите, что есть лишь одна возможная таблица умножения для группы из 2 элементов.
    \Sub{2} Покажите, что есть лишь одна возможная таблица умножения для группы из 3 элементов. 
    \Sub{4} Найдите все возможные таблицы умножения для группы из 4 элементов. 
\end{document}