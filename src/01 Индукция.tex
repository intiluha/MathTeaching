\documentclass[a4paper,12pt]{article}

\usepackage{school}

\begin{document}
    \head{7 сентября, 14 сентября}{Индукция}
    \rules{20 сентября}{10}
    
    {\bf Принцип математической индукции.} Пусть дана серия утверждений $P(n)$, по одному утверждению для каждого натурального числа. Пусть также для всех натуральных $k$ из утверждения $P(k)$ следует утверждение $P(k + 1)$. Тогда из $P(1)$ следует $P(n)$ для всех $n$.
    
    {\bf Игрушечный пример.} Пусть $P(n)$ это "сумма $n$ единиц положительна". Так как сумма $k + 1$ единицы это сумма суммы $k$ единиц с ещё одной единицей, а сумма двух положительных чисел положительна, то из $P(k)$ следует $P(k + 1)$. Так как единица положительна, то $P(1)$ верно. Значит, $P(n)$ верно для всех $n$. 
    
    \section{Задачи на разбор}
    
    \problem Используя индукцию, докажите, что следующие формулы верны.\\
    \sub $1 + 2 + \cdots + (n - 1) + n = \dfrac{n(n + 1)}{2}$\\
    \sub $1^2 + 2^2 + \cdots + (n - 1)^2 + n^2 = \dfrac{n(n + 1)(2n + 1)}{6}$\\
    \sub $1^3 + 2^3 + \cdots + (n - 1)^3 + n^3 = \left(\dfrac{n(n + 1)}{2}\right)^2$\\
    
    \problem Используя индукцию и соотношение $C_{n + 1}^{k + 1} = C_{n}^{k + 1} + C_{n}^{k}$, обоснуйте формулу бинома Ньютона:
    \begin{equation*}
        (a + b)^n = a^n + C_n^1 a^{n-1} b + C_n^2 a^{n-2} b^2 + \cdots + C_n^{n-1} a b^{n-1} +  b^n.
    \end{equation*}
    Из-за этой формулы, числа сочетаний $C_n^k$ также принято называть \textbf{биномиальными коэффициентами}. Часто можно встретить для них обозначение $\binom{n}{k}$. Обратите внимание, что в этих обозначениях $n$ сверху, а $k$ снизу.
    
    \problem Докажите, что для всех натуральных $n \geq 10$ выполнено неравенство $2^n~>~n^3$.
    
    \problem Несколько прямых делят плоскость на части. Докажите, что можно раскрасить эти части в чёрный и белый цвет так, чтобы части одного цвета не имели общих сторон.
    
    \problem Показать, что ханойскую башню из любого числа колец можно переложить на другой стержень, соблюдая правила игры.
    
    \problem Показать, что для любого $n > 2$ единицу можно представить как сумму $n$ различных дробей вида $\frac{1}{q}$.
    
    \section{Задачи для самостоятельного решения}
    
    \problem Показать, что любую сумму, начиная с 12 рублей, можно уплатить монетами в 3 рубля и 7 рублей.
    
    \problem На встрече некоторые люди пожали друг другу руки. \textbf{С помощью индукции} докажите, что число людей, сделавших нечётное число рукопожатий, чётно.
    
    \problem Пусть $n$ --- любое натуральное число. Показать, что квадрат $2^n \times 2^n$ с вырезанной угловой клеткой можно разрезать на уголки из трёх клеток.
    
    \problem На доске написаны 2 единицы. Каждую минуту между каждой парой соседних чисел на доске вписывают их сумму. 
    \begin{equation*}
        1\quad1 \quad\mapsto\quad 1\quad2\quad1 \quad\mapsto\quad 1\quad3\quad2\quad3\quad1 \quad\mapsto\quad 1\quad4\quad3\quad5\quad2\quad5\quad3\quad4\quad1 \quad\mapsto\quad \ldots
    \end{equation*}
    Найдите сумму чисел на доске через 100 минут.
    
    \problem Доказать, что для любого натурального $n$ выполнено неравенство
    \begin{equation*}
        1 + \frac{1}{2} + \frac{1}{4} + \frac{1}{8} + \ldots + \frac{1}{2^n} < 2.
    \end{equation*}
    
    \Problem{2} Доказать, что число $\underbrace{11 \ldots 11}_{3^n \text{единиц}}$ делится на $3^n$ для всех натуральных $n$.
    
    \Problem{2} На доске написано 1501 цифра --- нули и единицы (в любой комбинации и в любом порядке). Разрешается выполнять два вида действий:\\
    (1) менять первую цифру,\\
    (2) менять цифру, стоящую после первой единицы.\\
    Покажите, что комбинируя эти комбинации (в любом количестве) можно получить на доске любую последовательность.
    
    \Problem{3} Доказать, что для любого натурального $n$ выполнено неравенство
    \begin{equation*}
        1 + \frac{1}{4} + \frac{1}{9} + \frac{1}{16} + \ldots + \frac{1}{n^2} < 2.
    \end{equation*}
    
    \Problem{3} На краю пустыни имеется большой запас бензина и машина, которая при полной заправке может проехать 50 километров. Имеются (в неограниченном количестве) канистры, в которые можно сливать бензин из бензобака машины и оставлять на хранение (в любой точке пустыни). Доказать, что машина может проехать любое расстояние. (Канистры с бензином возить не разрешается, пустые можно возить в любом количестве.)
    
    % \begin{wrapfigure}{r}{0.2\linewidth}
    %     \vspace{-1ex}
    %     \begin{tikzpicture}
    %         \def \x {1.5}
    %         \def \y {1.5}
    %         \draw (0,0) node{} -- (3,0) node{} -- (4,2) node{} -- (1,2) node{} -- cycle;
    %         \filldraw[fill=green!20] (\x,\y) -- (0,0) -- (3,0) -- cycle -- (1,2) -- (4,2) -- cycle;
    %     \end{tikzpicture}
    % \end{wrapfigure}
\end{document} 