\documentclass[a4paper,12pt]{article}
\usepackage{school}

\begin{document}
    \head{11 января}{Снова индукция}
    \newrules
    
    \section{Задачи для самостоятельного решения}
    \problem Докажите, что $1 + 3 + 5 + 7 + \cdots + (2n - 1) = n^2$.
    \problem Докажите, что $(7^{2n} - 1) \isdiv 24$.
    \problem \textit{(Неравенство Бернулли)} Докажите, что $(1 + a)^n \geq 1 + an$ при $a > -1$ .
    \problem Докажите, что при каждом натуральном $n > 3$ существует выпуклый $n$-угольник, имеющий
ровно три острых угла.
    \problem Докажите, что $\left(1 - \dfrac{1}{4}\right)\left(1 - \dfrac{1}{9}\right) \cdots \left(1 - \dfrac{1}{n^2}\right) = \dfrac{n+1}{2n}$.
    \problem Последовательность натуральных чисел $a_n$ начинается с $a_0 = 0$ и $a_n = 3a_{n-1} + 2$.
Докажите, что $a_n = 3^n - 1$.
    \problem \sub $2^n > 2n + 1$ при натуральных $n > 2$. \\
    \sub $2^n > n^2$ при натуральных $n > 4$.
    \problem Докажите, что $(4^n + 15n - 1) \isdiv 9$.
    \problem \sub Докажите, что из $2^{n+1}$ натуральных чисел можно выбрать ровно $2^n$, сумма которых делится на $2^n$. \sub Докажите, что хватит и $2^{n+1} - 1$ числа.
    \problem На сколько частей делят плоскость $n$ прямых, если среди них нет параллельных и никакие три не пересекаются в одной точке?
    \problem Докажите, что $\dfrac{1 \cdot 3 \cdot 5 \cdot \cdots \cdot (2n - 1)}{2 \cdot 4 \cdot 6 \cdot \cdots \cdot 2n} < \dfrac{1}{\sqrt{3n + 1}}$.
    \problem На какое наибольшее число частей могут разбивать плоскость $n$ окружностей?
    
    \problem \sub Докажите, что сумма углов выпуклого шестиугольника равна $720^\circ$. \\
    \sub Найдите сумму углов выпуклого $n$-угольника.
    \problem Докажите, что сумма длин звеньев любой ломаной не может быть меньше, чем расстояние между началом ломаной и её концом.
    \problem Найдите ошибку в следующем рассуждении.
    \begin{solution}
        Докажем при помощи метода математической индукции, что любые $n$ карандашей имеют один и тот же цвет. При $n = 1$ доказывать нечего: карандаш один и цвет у него один. Предположим теперь, что при $n = k$ любые $n$ карандашей имеют один цвет, и докажем это утверждение для $n = k + 1$. Рассмотрим произвольный набор из $k + 1$ карандаша. Если мы отбросим последний карандаш, то по предположению индукции первые $k$ карандашей будут иметь один цвет. Если же мы отбросим первый карандаш, то по предположению индукции последние $k$ карандашей будут иметь один цвет. Значит, первый и последний карандаш одного цвета --- того самого, который имеют второй, третий, \ldots, $k$-ый карандаши. Итого, все карандаши одного цвета.
    \end{solution}
    \problem Шеренга новобранцев стоит перед старшиной. Старшина командует: \linebreak\mbox{«нале-ВО!»} Но по неопытности часть солдат поворачивается налево, а часть --- направо. После этого каждую секунду солдаты, оказавшиеся друг к другу лицом, понимают, что произошла ошибка, и оба поворачиваются кругом. Докажите, что, тем не менее, рано или поздно повороты прекратятся (при любом числе солдат и при любом их положении после команды старшины).
    \problem Докажите, что $\abs{x_1 + x_2 + \ldots + x_n} \leq \abs{x_1} + \abs{x_2} + \ldots + \abs{x_n}$ для любых вещественных $x_1, x_2, \ldots, x_n \in \R$. 
    \problem Докажите, что при $n > 2$ справедливо неравенство 
    \begin{equation*}
        \dfrac{1}{n + 1} + \dfrac{1}{n + 2} + \cdots \dfrac{1}{2n} < \dfrac{3}{5}.
    \end{equation*}
    \problem Докажите, что $2(\sqrt{n + 1} - 1) < \dfrac{1}{\sqrt{1}} + \dfrac{1}{\sqrt{2}} + \cdots + \dfrac{1}{\sqrt{n}} < 2 \sqrt{n}$ при всех $n$.
    \problem Бизнесмен заключил с чёртом сделку: он может любую имеющуюся у него купюру обменять у чёрта на любой набор купюр любого меньшего достоинства (по своему выбору, без ограничения общей суммы). Бизнесмен может также тратить деньги, но не может получать их в другом месте (кроме как у чёрта). При этом каждый день на еду ему нужен рубль. Сможет ли бизнесмен жить так бесконечно долго?
\end{document}