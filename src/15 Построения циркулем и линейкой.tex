\documentclass[a4paper,12pt]{article}
\usepackage{school}

\begin{document}
    \head{8 февраля}{Построения циркулем и линейкой}
    
    \section{Разбор}
    
    \defn \emph{Построением циркулем и линейкой} называется последовательность действий (любой длины) из следующего списка: \\ 
    \sub проведение прямой через 2 отмеченные точки; \\
    \sub проведение окружности с центром в отмеченной точке $A$ радиуса равного расстоянию между отмеченными точками $BC$; \\
    \sub отмечание точки пересечения двух прямых или двух окружностей или окружности и прямой.
    
    Во всех задачах этого листка, если не сказано обратного, требуется провести построение циркулем и линейкой, то есть найти эту самую последовательность действий. 
    \begin{remark}
        Можно и нужно пользоваться предыдущими задачами для сокращения работы. Представьте, что вы учите маленького ребёнка языку. Первые слова придётся объяснять жестами, но потом дело пойдёт куда быстрее --- ведь вы можете объяснить большую часть значения слова через уже известные. Увлекающимся программированием этот принцип хорошо известен.
    \end{remark}
    
    \problem Сумма двух отрезков $AB$ и $CD$.
    \problem Разность двух отрезков $AB$ и $CD$.
    \problem Серединный перпендикуляр к отрезку $AB$.
    \problem Угол от прямой $AB$, равный данному углу $\angle CDE$.
    \problem Середина данного отрезка $AB$.
    \problem Перпендикуляр к данной прямой $AB$ через заданную точку $C$.
    \problem Параллельная прямая к данной прямой $AB$ через заданную точку $C$.
    \problem Параллельный перенос отрезка $AB$ на заданный вектор $CD$.
    \problem Параллельный перенос отрезка $AB$ к заданной точке $C$.
    \problem Деление данного отрезка $AB$ на $n$ равных частей.
    \problem Биссектрисса данного угла $\angle ABC$.
    
    \section{Задачи для самостоятельного решения}
    
    \problem Треугольник по трём сторонам $AB$, $CD$ и $EF$.
    \problem Треугольник по двум сторонам $AB$, $CD$ и углу между ними $\angle EFG$.
    \problem Треугольник по стороне $AB$ и прилежащим углам $\angle CDE$, $\angle FGH$.
    \problem Треугольник по серединам сторон.
    \problem Треугольник по стороне и двум медианам, проведённым к другим сторонам.
    \problem Прямоугольный треугольник по гипотенузе и высоте, проведённой из прямого угла.
    \problem Касательная к окружности через данную точку вне окружности.
    \problem Дан угол, равный $19^\circ$. Разделите его на 19 равных частей.
    \problem Внутри данного угла отмечена точка. Проведите через эту точку прямую так, чтобы её отрезок, заключённый внутри угла, делился этой точкой пополам.
    \problem Трапеция \sub по основаниям $AB$, $CD$ и боковым сторонам $EF$, $GH$; \sub по основаниям $AB$, $CD$ и диагоналям $EF$, $GH$.
    \problem Треугольник по периметру и двум углам.
    \problem Пятиугольник по серединам сторон.
    \problem Квадрат по четырём точкам, лежащим начетырёх его сторонах.
    
\end{document}