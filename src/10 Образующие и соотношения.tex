\documentclass[a4paper,12pt]{article}
\usepackage{school}

\begin{document}
    \head{12 декабря}{Образующие и соотношения}
    
    \section{Разбор}
    
    \defn Набор элементов $g_1, \ldots, g_k \in G$ называют \href{https://ru.wikipedia.org/wiki/\%D0\%9F\%D0\%BE\%D1\%80\%D0\%BE\%D0\%B6\%D0\%B4\%D0\%B0\%D1\%8E\%D1\%89\%D0\%B5\%D0\%B5_\%D0\%BC\%D0\%BD\%D0\%BE\%D0\%B6\%D0\%B5\%D1\%81\%D1\%82\%D0\%B2\%D0\%BE_\%D0\%B3\%D1\%80\%D1\%83\%D0\%BF\%D0\%BF\%D1\%8B}{набором \emph{образующих} группы} $G$, если любой элемент группы может быть записан в виде произведения нескольких $g_i$ и обратных к ним (то есть разрешены выражения вроде $g_1g_2g_2g_1^{-1}g_3g_4^{-1}$).
    \example \href{https://ru.wikipedia.org/wiki/\%D0\%94\%D0\%B8\%D1\%8D\%D0\%B4\%D1\%80\%D0\%B0\%D0\%BB\%D1\%8C\%D0\%BD\%D0\%B0\%D1\%8F_\%D0\%B3\%D1\%80\%D1\%83\%D0\%BF\%D0\%BF\%D0\%B0}{Группу диэдра} $D_3$ (то есть группу симметрий правильного треугольника) можно породить любой осевой симметрией $s$ и любым поворотом $r$. В этих обозначениях $D_3 = \{e, r, r^2, s, rs, r^2s\}$.
    \problem Есть ли другие двухэлементные наборы образующих $D_3$?
\begin{solution}
    Да, любая пара осевых симметрий тоже порождает $D_3$. Например, через $rs$ и $r^2s$ можно выразить $r$ и $s$ следующим образом:
    \begin{align*}
        (r^2s)(rs)^{-1} = r^2ss^{-1}r^{-1} = r^2r^{-1} &= r \\
        ((r^2s)(rs)^{-1})(r^2s) = rr^2s &= s.
    \end{align*}
\end{solution}
    \example Группу целых чисел по сложению $\Z$ можно породить всего одним элементом --- единицей. $-1$ тоже подойдёт.
    \problem \sub Найдите трёхэлементный набор образующих для $S_3$. \\
    \sub Найдите двухэлементный набор образующих для $S_3$. \\
    \sub Можно ли обойтись одной образующей?
    \begin{solution}
        \sub Согласно теореме из листка 07, любая перестановка может быть записана, как произведение транспозиций, так что в качестве образующих можно взять $\cycle{1,2}$, $\cycle{2,3}$ и $\cycle{1,3}$. \\
        \sub Согласно задачке из того же листка, можно обойтись только элементарными транспозициями, так что в качестве образующих можно взять $\cycle{1,2}$ и $\cycle{2,3}$. \\
        \sub Если бы можно было обойтись одной образующей $x$, то все элементы группы были бы степенями $x$, а значит порядок $x$ был бы равен 6. Но в $D_3$ все элементы имеют порядок 1, 2 или 3, так что это невозможно.
    \end{solution}
    \problem \sub Найдите трёхэлементный набор образующих для $S_4$. \\
    \sub Найдите двухэлементный набор образующих для $S_4$.
    \begin{solution}
        \sub Аналогично предыдущей задаче, можно взять элементарные транспозиции $\cycle{1,2}$, $\cycle{2,3}$ и $\cycle{3,4}$. \\
        \sub Возьмём элементы $x = \cycle{1,2,3,4}$ и $\cycle{1,2}$. Несложно посчитать, что $x\cycle{1,2}x^{-1} = \cycle{2,3}$ и $x\cycle{2,3}x^{-1} = \cycle{3,4}$. 
    \end{solution}
    
    \defn \emph{Группой остатков по модулю $n$} или \emph{группой вычетов по модулю $n$} называется множество $\Z/n\Z$ остатков по модулю $n$ с операцией <<сложение по модулю $n$>>. Элемент этой группы, соответствующий остатку от числа $k \in \Z$ записывают как $[k]_n$.
    \example Следующая таблица описывает операцию в $\Z/6\Z$.
    \begin{center}
        \begin{equation*}
        \newcommand{\tab}{\phantom{r_{240}}}
            \begin{array}{c||c|c|c|c|c|c}
                +     & [0]_6 & [1]_6 & [2]_6 & [3]_6 & [4]_6 & [5]_6 \\ \hline \hline
                [0]_6 & [0]_6 & [1]_6 & [2]_6 & [3]_6 & [4]_6 & [5]_6 \\ \hline
                [1]_6 & [1]_6 & [2]_6 & [3]_6 & [4]_6 & [5]_6 & [0]_6 \\ \hline
                [2]_6 & [2]_6 & [3]_6 & [4]_6 & [5]_6 & [0]_6 & [1]_6 \\ \hline
                [3]_6 & [3]_6 & [4]_6 & [5]_6 & [0]_6 & [1]_6 & [2]_6 \\ \hline
                [4]_6 & [4]_6 & [5]_6 & [0]_6 & [1]_6 & [2]_6 & [3]_6 \\ \hline
                [5]_6 & [5]_6 & [0]_6 & [1]_6 & [2]_6 & [3]_6 & [4]_6
            \end{array}
        \end{equation*}
    \end{center}
    \problem \sub Каков минимальный набор образующих для $\Z/6\Z$? \sub А для $\Z/n\Z$?
    \begin{solution}
        Минимальный набор содержит всего одну образующую --- $[1]_n$.
    \end{solution}
    \defn Группа называется \emph{циклической}, если у неё существует набор из одной образующей.
    \example Группа $\Z$ и все группы вычетов --- циклические. На самом деле это полный список циклических групп, но чтобы это доказать, сперва нужно определить, какие группы мы считаем одинаковыми. Мы сделаем это через несколько недель.
    
    \defn \emph{Соотношениями} называются равенства разных выражений от образующих группы. Пример: $g_1g_2 = g_2g_1$. Так как в группе у любого элемента есть обратный, всякое соотношение можно записать в виде $(\ldots) = e$. Соотношение из примера выше запишется как $g_1^{-1}g_2^{-1}g_1g_2 = e$.
    \example \sub Группа $\Z$ не имеет соотношений. \\
    \sub В группе $\Z/n\Z$ с образующей $[1]_n$ выполнено соотношение $\underbrace{[1]_n + \ldots + [1]_n}_{n} = e$. \\ Все остальные соотношения в $\Z/n\Z$ следуют из этого, например $\underbrace{[1]_n + \ldots + [1]_n}_{3n} = e$. \\
    \sub В группе $D_3$ выполнены соотношения $srsr = e$, $s^2 = e$ и $r^3 = e$.
    
    \defn Говорят, что группа $G$ \emph{задана образующими $g_1, \ldots, g_k \in G$ и соотношениями $r_1, \ldots, r_l$}, если $g_1, \ldots, g_k$ порождают $G$, а все соотношения между ними <<следуют>> из $r_1, \ldots, r_l$.
    \problem Докажите, что \\ 
    \sub Группа $\Z$ порождена одной образующей $1$ без соотношений. \\
    \sub Группа $\Z/n\Z$ порождена одной образующей $[1]_n$ с одним соотношением. \\
    \sub Группа $D_3$ порождена двумя образующими $r, s$ с тремя соотношениями.
    \begin{solution}
        \sub\hspace{-1.5mm}--\sub Очевидно. \\
        \sub Из соотношения $srsr = e$ можно получить $sr = r^{-1}s^{-1}$. С учётом остальных двух соотношений, получаем $sr = r^2s$. Используя это соотношение, любое слово из букв $r$ и $s$ можно <<причесать>>, то есть привести к виду $r^\alpha s^\beta$. А с учётом того, что $r^3 = s^2 = e$, можно считать $0 \leq \alpha \leq 2$ и $0 \leq \beta \leq 1$. Таких слов уже всего 6, как и элементов в нашей группе, так что дополнительных соотношений не требуется. 
    \end{solution}
    
    \section{Задачи для самостоятельного решения}
    
    \problem \sub Найдите двухэлементный набор образующих для $D_4$; \Sub{0} для $D_5$; \sub для $D_n$. \\
    \sub Найдите $n-1$-элементный набор образующих для $S_n$. \\
    \Sub{2} Найдите $2$-элементный набор образующих для $S_n$.
    
    \problem Рассмотрим множество пар целых чисел $G = \Z \times \Z$ с покомпонентным сложением $(x_1, y_1) + (x_2, y_2) = (x_1 + x_2, y_1 + y_2)$ в качестве операции. \\
    \sub Покажите, что $G$ является группой. \\
    \sub Найдите образующие для $G$. \\
    \sub Найдите соотношения для $G$.
    
    \problem \sub Группа $S_3$ порождена образующими $x = \cycle{1,2}$ и $y = \cycle{2,3}$. Найдите три независимых соотношения на эти образующие. \\
    \Sub{2} Задайте группу $S_3$ образующими и соотношениями. \\
    \Sub{4} Задайте группу $S_4$ образующими и соотношениями.
\end{document}