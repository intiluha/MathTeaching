\documentclass[a4paper,12pt]{article}

\usepackage{school}

\begin{document}
    \head{19 октября}{Отображения}
    \rules{25 октября}{15}
    
    \section{Задачи на разбор}
    
    \defn \emph{Отображение} $f$ множества $X$ в множество $Y$ --- правило, сопоставляющее \textit{каждому} элементу $x \in X$ \textit{ровно один} элемент $f(y) \in Y$. Синоним: функция $f$ из $X$ в $Y$. Обозначение: $f \colon X \to Y$.
    
    \example %     inj,     surj
    $X = Y$ --- любое множество, $f(x) = x$. Такое отображение называется \emph{тождественным} и обозначается $\mathds{1}_X$.
    \example \label{pupil_school} % not inj,     surj
    $X = \{\text{московские школьники}\}$, $Y = \{\text{школы в Москве}\}$, \\
    $f(x) = \text{школа, в которой учится } x$.
    \example %     inj,     surj
    $X = \{\text{точки плоскости}\}$, $Y = \{\text{точки плоскости}\}$, \\
    $f = \text{поворот относительно начала координат на угол }37^\circ$.
    \example % not inj, not surj
    $X = \{\text{вещественные числа}\}$, $Y = \{\text{вещественные числа}\}$, $f(x) = x^2$.
    \example \label{natural_to_real_inclusion} %     inj, not surj
    $X = \{\text{натуральные числа}\}$, $Y = \{\text{вещественные числа}\}$, $f(x) = x$.
    \example % ??? inj, not surj
    $X = \{\text{слушатели нашего кружка}\}$, $Y = \{\text{вещественные числа}\}$, \\
    $f(x) = \text{рейтинг } x$.
    
    \defn Отображение $f \colon X \to Y$ называется:\\
    \emph{сюръективным}, если в каждый элемент кто-то переходит --- 
    \begin{equation*}
        \forall y \in Y (\exists x \in X (f(x) = y));
    \end{equation*}
    \emph{инъективным}, если разные элементы переходят в разные --- 
    \begin{equation*}
        \forall x_1, x_2 \in X (x_1 \neq x_2 \Rightarrow f(x_1) \neq f(x_2));
    \end{equation*}
    \emph{биективным} или \emph{взаимно-однозначным}, если выполнено и то, и другое.
    
    \problem Какие из отображений в примерах \sub сюръективны; \sub инъективны; \sub биективны?
\begin{solution}
    \setcounter{example}{0}
    \example $f$ биективно.
    \example $f$ сюръективно, но не инъективно. В самом деле, в каждой московской школе есть хотя бы один ученик, причём в некоторых из них (скорее всего, во всех) учится больше одного ученика. 
    \example $f$ биективно. Это частный случай задачки из позапрошлого листка по движениям плоскости.
    \example $f$ не сюръективно и не инъективно. В самом деле, в отрицательные числа никто не переходит (так как квадрат числа всегда положительный) и квадраты разных чисел могут быть одинаковыми ($2^2 = (-2)^2$).
    \example $f$ инъективно, но не сюръективно. В самом деле, разные натуральные числа тавтологически являются разными действительными числами. Однако, не всякое действительное число является натуральным.
    \example $f$ не инъективно и не сюръективно. В самом деле, на момент написания этого текста, у Семёна и Ивана одинаковый рейтинг ($28\%$), значит $f$ не инъективно. Рейтинга $100500\%$ ни у кого нет, значит $f$ не сюръективно.
\end{solution}
    
    \defn Композиция отображений $f \colon X \to Y$ и $g \colon Y \to Z$ --- отображение (обратите внимание на поядок!) $g \circ f \colon X \to Z$, определённое следующей формулой
    \begin{equation*}
        \forall x \in X \quad (g \circ f)(x) = g(f(x)).
    \end{equation*}
    
    \problem Докажите, что композиция отображений ассоциативна.
    \begin{solution}
        Напомним, что ассоциативность означает $f_1 \circ (f_2 \circ f_3) = (f_1 \circ f_2) \circ f_3$. Чтобы показать равенство отображений, нужно показать что они действуют одинаково на каждый элемент. Для произвольного $x$ имеем
        \begin{equation*}
            (f_1 \circ (f_2 \circ f_3))(x) = f_1((f_2 \circ f_3)(x)) = f_1(f_2(f_3(x))) = ((f_1 \circ f_2)(f_3(x)) = ((f_1 \circ f_2) \circ f_3)(x),
        \end{equation*}
        что и требовалось.
    \end{solution}
    \problem Правда ли, что композиция двух биекций --- биекция?
    \begin{solution}
        Пока без комментариев :)
    \end{solution}
    
    \defn Отображение $g \colon Y \to X$ называется \emph{правым обратным} к $f \colon X \to Y$, если $f \circ g = \mathds{1}_Y$. Отображение $g \colon Y \to X$ называется \emph{левым обратным} к $f \colon X \to Y$, если $g \circ f = \mathds{1}_X$.
    
    \problem Предъявите пример \sub правого обратного к отображению из примера \ref{pupil_school}; \sub левого обратного к отображению из примера \ref{natural_to_real_inclusion}.
    \begin{solution}
        \sub Достаточно взять любое отображение из $Y$ в $X$, переводящее каждую школу в какого-нибудь ученика этой школы. Например, первого по алфавиту в журнале или седьмого по росту.
        \sub Достаточно взять любое отображение из вещественных чисел в натуральные, не меняющее натуральные числа. Например:
        \begin{equation*}
            g(y) = 
            \begin{cases}
                y, & \text{если $y$ --- натуральное} \\
                1, & \text{иначе}
            \end{cases}
        \end{equation*}
    \end{solution}
    \problem Покажите, что всякое сюръективное отображение имеет правое обратное.
    \begin{solution}
        Хотим построить правое обратное к отображению $f \colon X \to Y$. Для любого $y \in Y$ обозначим через $f^{-1}(y)$ \emph{слой отображения} $f$ над $y$, то есть множество всех таких $x \in X$, что $f(x) = y$. Например, слои отображения из примера \ref{pupil_school} --- множества школьников, обучающихся в одной конкретной школе. Сюръективность отображения $f$ обозначает, что все наши слои непусты. Теперь выберем любым образом по элементу $x \in f^{-1}(y)$ в каждом таком слое и положим $g(y) = x$. Это задаст нам некоторое отображение $g \colon Y \to X$. Легко видеть, что оно правое обратное, как и требовалось. \\
        \textit{Для любознательных: для бесконечных множеств этот выбор по элементу в каждом слое не так безобиден, как может показаться. Возможность этого выбора постулируется в математике отдельной аксиомой --- аксиомой выбора.}
    \end{solution}
    \problem Покажите, что всякое не инъективное отображение не имеет левого обратного. (Другими словами: если у отображения $f$ есть левое обратное, то $f$ инъективно).
    \begin{solution}
        От противного: пусть у $f \colon X \to Y$ есть левое обратное $g \colon Y \to X$, но $f$ не инъективно. Тогда есть два разных элемента $x_1, x_2 \in X$, такие что $f(x_1) = f(x_2)$. Тогда имеем
        \begin{equation*}
            x_1 = \mathds{1}_X(x_1) = g(f(x_1)) = g(f(x_2)) = \mathds{1}_X(x_2) = x_2,
        \end{equation*}
        что противоречит предположению.
    \end{solution}
    
    \section{Задачи для самостоятельного решения}
    
    \problem Пусть $\abs{X} = n$. Сколько существует биекций $f \colon X \to X$?
    \problem Пусть $\abs{X} = n, \abs{Y} = k$. \\ 
    \sub Сколько различных отображений $f \colon X \to Y$ существует? \\
    \sub Сколько из них инъективных? \\
    \Sub{3} Сколько из них сюръективных?
    \problem Правда ли, что композиция отображений коммутативна?
    \problem Правда ли, что: \\
    \sub композиция двух сюръекций --- сюръекция? \\
    \sub композиция двух инъекций --- инъекция? \\
    \sub композиция двух биекций --- биекция?
    \problem Покажите, что отображение: \\
    \Sub{2} сюръективно тогда и только тогда, когда имеет правое обратное; \\
    \Sub{2} инъективно тогда и только тогда, когда имеет левое обратное.
    \problem Пусть отображение $f$ имеет правое обратное $g$ и левое обратное $h$. Докажите, что \sub $g = h$ и \sub $f$ биективно.
    \problem Пусть $X = \{1, 2, 3, 4\}$ в $Y = \{a, b\}$. \\
    \sub Опишите все отображения из $X$ в $Y$ и посчитайте, сколько правых обратных у каждого из них. \\
    \sub Опишите все отображения из $Y$ в $X$ и посчитайте, сколько левых обратных у каждого из них.
    \problem Пусть $\R$ --- множество вещественных чисел. Какие из следующих отображений $f \colon \R \to \R$ инъективны? Сюръективны? \\
    \sub $f(x) = x + 7$; \\
    \sub $f(x) = 3x - 5$; \\
    \sub $f(x) = \abs{x}$; \\
    \sub $f(x) = x^3$; \\
    \Sub{2} $f(x) = x^3 - x$; \\
    \Sub{2} $f(x) = 2^x$.
\end{document} 