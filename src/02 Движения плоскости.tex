\documentclass[a4paper,12pt]{article}

\usepackage{school}

\begin{document}
    \head{21 сентября}{Движения плоскости}
    \rules{27 сентября}{12}
    
\begin{define}
    Движением называется любое преобразование плоскости, сохраняющее расстояния. Композицией движений называется последовательное их применение. Обозначение --- $F_1 \circ F_2$.
    \end{define}
    
    \textbf{Игрушечный пример.} Тождетвенное преобразование $e$ (то есть оставляющее все точки плоскости на месте) является движением.
    
    \textbf{Пример.} Все осевые симметрии, повороты и параллельные переносы являются движениями.
    
    \section{Задачи на разбор}
    
    \problem Покажите, что композиция движений является движением.
    
    \problem Покажите, что композиция движений ассоциативна:
    \begin{equation*}
        F_1 \circ (F_2 \circ F_3) = (F_1 \circ F_2) \circ F_3.
    \end{equation*}
    
    \section{Задачи для самостоятельного решения}
    
    \problem Верно ли, что композиция движений коммутативна (перестановочна): 
    \begin{equation*}
        F_1 \circ F_2 = F_2 \circ F_1?
    \end{equation*}
    Докажите или приведите контрпример.
    
    \Problem{0.5} Зафиксированы точки $A$ и $A'$. Покажите, что существует осевая симметрия, переводящая $A$ в $A'$.
    
    \Problem{0.5} Зафиксированы точки $A$, $B$ и $B'$, причём $AB = AB'$. Покажите, что существует осевая симметрия, оставляющая $A$ на месте и переводящая $B$ в $B'$.
    
    \Problem{0.5} Зафиксированы точки $A$, $B$, $C$ и $C'$, причём $AC = AC'$ и $BC = BC'$. Покажите, что существует осевая симметрия, оставляющая $A$ и $B$ на месте и переводящая $C$ в $C'$.
    
    \problem Зафиксируем 3 точки плоскости $A, B, C$, не лежащие на одной прямой. Покажите, что единственное движение, оставляющее их все на месте --- тождественное.
    
    \problem Зафиксируем 3 точки плоскости $A, B, C$, не лежащие на одной прямой. Зафиксируем ещё 3 точки плоскости $A', B', C'$, также не лежащие на одной прямой. Пусть $AB = A'B'$, $BC = B'C'$ и $CA = C'A'$. Покажите, что \\
    \sub существует движение, переводящее $A$ в $A'$, $B$ в $B'$ и $C$ в $C'$; \\
    \Sub{2} такое движение единственно.
    
    \problem Покажите, что любое движение является композицией осевых симметрий, причём достаточно использовать не более трёх штук.
    
    \problem Покажите, что любое движение \\
    \Sub{0.5} переводит разные точки в разные; \\
    \sub взаимно-однозначно (в каждую точку переходит ровно одна точка).
    
    \problem Покажите, что для любого движения $F$ есть единственное \textbf{обратное} движение $F^{-1}$, то есть такое, что $F \circ F^{-1} = F^{-1} \circ F = e$.
    
    \problem Покажите, как представить \\
    \sub параллельный перенос на любой вектор; \\
    \sub поворот вокруг любой точки на любой угол \\
    композицией двух осевых симметрий.
    
    \Problem{2} \label{glide} Приведите пример движения, не являющегося осевой симметрией, поворотом или параллельным переносом. Придумайте название для движений такого типа. \textit{Формальная часть задачи --- в первом предложении. Тем не менее попробуйте разобраться, как можно описать ваш пример более общо, выделить его в какую-то группу. Если не получается, я расскажу.}
    
    \Problem{2} Докажите \textbf{теорему Шаля}: любое движение является осевой симметрией, поворотом, параллельным переносом или движением типа найденного в задаче \ref{glide}.
    
    \Problem{4} Составьте <<таблицу умножения>> для движений. В каждую клетку должны быть вписаны все типы движений, которые можно получить. Звёздочкой $*$ обозначено движение из задачи \ref{glide}. Можно получить баллы за заполнение части таблицы.
    \begin{center}
        \newcommand{\tab}{\phantom{ }}
        \begin{tabular}{p{0.18\textwidth}||p{0.18\textwidth}|p{0.18\textwidth}|p{0.18\textwidth}|p{0.18\textwidth}}
            \tab & осевая симметрия & поворот & \Above{параллельный\\перенос} & $*$ \\ \hline \hline
            \Above{осевая\\симметрия} & \tab & \tab & \tab & \tab \\ \hline 
            поворот &&&& \\ \hline 
            \Above{параллельный\\перенос} &&&& \\ \hline 
            $*$ &&&&
        \end{tabular}
    \end{center}

\end{document} 