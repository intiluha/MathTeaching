\documentclass[a4paper,12pt]{article}
\usepackage{school}

\begin{document}
    \head{12 апреля}{Бинарные отношения}
    \iirules
    
    \section{Разбор}
    \defn \emph{Бинарным отношением} на множестве $X$ называется отношение, определённое на парах элементов из $X$. Более формально, бинарным отношением называется любое подмножество $R \subset X \times X$ множества пар элементов из $X$. Утверждение <<$x$ и $y$ состоят в отношении $R$>> (формально: $(x,y) \in R$) записывают как $xRy$. Для конечного множества, отношение можно записать в виде ориентированного графа, вершины которого --- элементы множества.
    \example \label{RealationsExample} \sub Отношение $>$ на множестве $\{1,2,3\}$. \\
    \sub Отношение <<быть тёзкой>> на множестве учеников 8-4. \\
    \sub Отношение $=$ на множестве $\{1,2,3\}$. \\
    \sub Отношение подобия на множестве треугольников на плоскости. \\
    \sub Отношение соседства на множестве граней куба. \\
    \sub Отношение делимости на множестве $\Z$.
    \problem Сколько всего есть бинарных отношений на множестве \sub из двух; \sub из трёх элементов?
    
    \defn Бинарное отношение $R$ на множестве $X$ называется
    \begin{itemize}
        \item \emph{рефлексивным}, если для любого $x \in X$ выполнено $xRx$;
        \item \emph{антирефлексивным}, если для любого $x \in X$ \textit{не} выполнено $xRx$;
        \item \emph{симметричным}, если для любых $x, y \in X$ из $xRy$ следует $yRx$;
        \item \emph{антисимметричным}, если для любых $x, y \in X$ из $xRy$ и $yRx$ следует $x=y$;
        \item \emph{транзитивным}, если для любых $x, y, z \in X$ из $xRy$ и $yRz$ следует $xRz$.
    \end{itemize}
    \problem Какие из отношений из примера \ref{RealationsExample} являются \sub рефлексивными; \sub антирефлексивными; \sub симметричными; \sub антисимметричными; \sub транзитивными?
    \problem Приведите пример отношения на множестве $\{1,2,3,4\}$, которое является только рефлексивным (не антирефлексивным, не симметричным, не антисимметричным, не транзитивным).
    
    \defn \emph{Отношением эквивалентности} называется рефлексивное, симметричное и транзитивное бинарное отношение.
    \defn \emph{Отношением нестрогого частичного порядка} называется рефлексивное, антисимметричное и транзитивное бинарное отношение. \emph{Отношением строгого частичного порядка} называется антирефлексивное, антисимметричное и транзитивное бинарное отношение.
    
    \section{Задачи для самостоятельного решения}
    \problemw Выпишите все пары, входящие \\
    \sub в отношение $=$ на множестве $\{1,2,3,4\}$; \\
    \sub в отношение $\leq$ на множестве $\{1,2,3,4\}$; \\
    \sub в отношение делимости на множестве $\{1,2,3,4,5,6\}$; \\
    \sub в отношение соседства на множестве граней куба (обозначьте их удобным вам способом).
    \problemw Сколько всего есть бинарных отношений на множестве \sub из четырёх; \Sub{2} из $n$ элементов?
    
    \problem Приведите пример отношения на множестве $\{1,2,3,4\}$, которое является \sub только антирефлексивным; \sub только симметричным; \sub только антисимметричным; \sub только транзитивным.
    
    \problem Какие из отношений из примера \ref{RealationsExample} являются \sub отношениями эквивалентности; \sub нестрогими порядками; \sub строгими порядками?
    \problemw Сколько всего есть отношений эквивалентности на множестве \sub из двух; \sub из трёх; \Sub{2} из четырёх элементов?
    \problem Пусть множество $X$ разбито на несколько подмножеств $X_i \subset X$, то есть $X_i$ не пересекаются, но в объединении дают весь $X$. Определим бинарное отношение~$\sim$ на множестве $X$ следующим образом: $x \sim y$ если и только если $x$ и $y$ лежат в одном куске, то есть $\exists i$, такое что $x, y \in X_i$. Покажите, что $\sim$ является отношением эквивалентности.
\end{document}