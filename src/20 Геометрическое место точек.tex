\documentclass[a4paper,12pt]{article}
\usepackage{school}

\begin{document}
    \head{3 апреля}{Геометрическое место точек}
    \iirules
    
    \section{Разбор}
    \problem \sub Найдите ГМТ, равноудаленных от двух точек $A$ и $B$. \sub Найдите ГМТ, которые ближе к точке $A$, чем к точке $B$.
    \problem Найдите геометрическое место центров окружностей, касающихся данной прямой и проходящих через данную точку.
    \begin{solution}
        Введём на плоскости координаты таким образом, что 
    \end{solution}
    \problem Найти ГМТ, из которых данный отрезок виден под углом $90^\circ$.
    
    \section{Задачи для самостоятельного решения}
    \problem Найдите ГМТ равноудалённых о двух заданных прямых.
    \problem Даны две точки $A$ и $B$. Найдите геометрическое место точек $M$ таких, что $AM + MB = AB$.
    \problem Дан отрезок $AB$. Найдите ГМТ $M$ таких, что $AM$ --- наименьшая сторона треугольника $ABM$.
    \problem Дан четырехугольник $ABCD$. Оказалось, что существуют две такие точки $O$, что $AO = DO$ и $BO = CO$. Докажите, что $AD$ и $BC$ параллельны.
    \problem Дан \textit{прямоугольник} $ABCD$. Найдите ГМТ $X$, для которых $AX + BX = CX + DX$.
    \problem Даны два отрезка $AB$ и $CD$ равной длины. Сколько существует точек $M$ таких, что треугольники $ABM$ и $CDM$ равны?
    \problem На сторонах $AB$ и $BC$ треугольника $ABC$ берутся точки $D$ и $E$ (по одной на каждой). Найдите геометрическое место середин отрезков $DE$.
    \Problem{2} Диагонали четырехугольника равны. Известно, что серединный перпендикуляр к одной его стороне пересекает противоположную сторону. Докажите, что это верно и для противоположной стороны.
    \Problem{2} Дан шестиугольник, никакие стороны которого не параллельны. Стороны покрашены в черный и белый цвет по очереди. Сколько существует точек, которые равноудалены от всех черных сторон?
    \Problem{2} Найдите ГМТ $M$, лежащих внутри ромба $ABCD$ и обладающих тем свойством, что $\angle AMD + \angle BMC = 180^\circ$.
\end{document}