\documentclass[a4paper,12pt]{article}
\usepackage{school}

\begin{document}
    \head{28 декабря}{Математическая Регата}
    
    \section{Первый тур (10 минут; каждая задача --- 6 баллов)}
    \problem Найдите значение выражения
    \begin{equation*}
        \left(1 + \frac{1}{2}\right) \left(1 - \frac{1}{3}\right) \left(1 + \frac{1}{4}\right) \left(1 - \frac{1}{5}\right) \cdots \left(1 + \frac{1}{2m}\right) \left(1 - \frac{1}{2m + 1}\right).
    \end{equation*}
    \problem Существует ли параллелограмм, который можно разрезать на семь попарно неравных трапеций?
    \problem На какую цифру нужно заменить $*$ так, чтобы разность $98765 - 4321*$ была кратна 12?
    
    \section{Второй тур (15 минут; каждая задача --- 7 баллов)}
    \problem Каждый цветок на поляне цветет ровно 30 дней. Известно, что каждый день пять цветков увядают, а взамен распускаются пять новых. Сколько цветущих цветков на поляне?
    \problem $O$ --- центр квадрата $ABCD$. Точка $P$ внутри квадрата такова, что треугольник $APD$ --- равносторонний. $M$ и $N$ --- середины отрезков $BP$ и $CP$. Докажите, что треугольник $MON$ --- также равносторонний.
    \problem В десяти лунках, расположенных по кругу, лежат 55 камней. В любых двух лунках разное количество камней и пустых лунок нет. Докажите, что найдутся три лунки, стоящие подряд, в которых в сумме меньше, чем 16 камней.
    
    \section{Третий тур (20 минут; каждая задача --- 8 баллов)}
    \problem Про положительные числа $a$, $b$ и $c$ известно, что $a^2 < b$, $b^2 < c$, $c^2 < a$. Докажите, что каждое их этих чисел меньше единицы.
    \problem Биссектриса $CF$ и высота $BH$ треугольника $ABC$ пересекаются в точке $O$. Найдите углы треугольника, если $CO$ = $OF$ и $BO = 2OH$.
    \problem В семье шестеро детей. Пятеро из них старше самого младшего на 2, 6, 8, 12 и 14 лет соответственно. Сколько лет младшему, если возрасты всех детей – простые числа?
    
    \section{Четвертый тур (25 минут; каждая задача --- 9 баллов)}
    \problem Можно ли, используя только сложение и вычитание, получить ноль из квадратов первых двухсот натуральных чисел?
    \problem Угол $B$ ромба $ABCD$ равен $40^\circ$, $E$ --- середина $BC$, $F$ --- основание перпендикуляра, опущенного из $A$ на $DE$. Найдите угол $DFC$.
    \problem В однокруговом футбольном турнире играют восемь команд, четыре из которых выходят в финал. При равенстве очков, проходящие в финал команды определяется жребием. Какое наименьшее количество очков гарантирует выход в финал? (Победа --- 3 очка, ничья --- 1 очко, поражение --- 0.)
\end{document}