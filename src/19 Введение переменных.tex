\documentclass[a4paper,12pt]{article}
\usepackage{school}

\begin{document}
    \head{29 марта}{Введение переменных}
    \iirules
    
    \problemw Мальвина велела Буратино разделить число на 2, а к результату прибавить 3. Он же по ошибке умножил число на 2, а от полученного произведения отнял 3. Но ответ все равно получился правильный. Какой?
    \problemw Один градус шкалы Цельсия равен $1.8$ градусов шкалы Фаренгейта, при этом $0^\circ$ по Цельсию соответствует $32^\circ$ по шкале Фаренгейта. Может ли температура выражаться одинаковым числом градусов как по Цельсию, так и по Фаренгейту? 
    \problem На окружности отметили 100 точек и соединили каждые две отрезком. Точки покрасили в два цвета. Какое наибольшее количество отрезков с концами в точках разного цвета могло получиться?
    \illustrated{[scale=0.5]
        \newcommand{\x}{2.7}
        \draw (0,0) -- ++(0,\x+3) -- ++(2*\x+5,0) -- ++(0,-2*\x-3) -- ++(-2*\x-1,0) +(0,\x) -- ++(0,\x-4) -- ++(-4,0) -- ++(0,4) -- ++(\x+4,0) ++(0,-\x) -- ++(0,\x+1) +(\x+1,0) -- ++(-1,0) +(0,-1) -- ++(0,\x+2); 
    }{
        \problemw Фигура на рисунке составлена из квадратов. Найдите сторону левого нижнего квадрата, если известно, что сторона самого маленького квадрата равна 1.
    }
    \problemw В турнире Солнечного города по шахматам каждый из 100 участников сыграл с каждым ровно по одному разу (турнир в один круг). После турнира Незнайка неожиданно узнал, что за победу действительно давалось 1 очко, но за ничью давалось не $\dfrac{1}{2}$ очка, как он думал, а 0 очков, а за поражение --- не 0 очков, а $-1$. В результате Незнайка набрал в два раза меньше очков, чем ему казалось. Сколько очков набрал Незнайка?
    \illustrated{[scale=0.5, center/.style={inner sep=0pt, minimum size=2mm, draw=black, fill=black, circle}]
        \newcommand{\x}{6}
        \newcommand{\mycircle}[2]{\draw (#1,0) node[center] {} circle[radius=#2];}
        \draw (-\x,0) -- (\x,0);
        \mycircle{0}{\x} \mycircle{\x/2}{\x/2} \mycircle{\x/2}{\x/2-1} \mycircle{\x-1}{1} \mycircle{1}{\x-3} \mycircle{2-\x}{2}
    }{
        \problem Шесть окружностей расположили на плоскости как на рисунке (центры окружностей отмечены, все они лежат на одной прямой). Известно, что диаметр правой (самой маленькой) окружности равен $2$. Какой радиус имеет самая левая из внутренних окружностей?
    }
    \problem Лестница, стоявшая на гладком полу у стены, соскальзывает вниз (все время касаясь стены). По какой линии движется котенок, сидящий на середине лестницы?
    \Problem2 В турнире по волейболу несколько команд сыграли в один круг (каждая играла с каждой по одному разу, ничьих в волейболе не бывает). Пусть $P$ --- сумма квадратов чисел, задающих количество побед каждой команды, $Q$ --- сумма квадратов чисел, задающих количество их поражений. Докажите, что $P = Q$.
\end{document}