\documentclass[a4paper,12pt]{article}

\usepackage{school}

\begin{document}
    \head{16 ноября}{Перестановки-2}
    
    \section{Разбор}
    
    \defn Пусть $\sigma \in S_n$ --- перестановка. \emph{Подгруппой, порождённой $\sigma$} называется множество всех степеней $\sigma$, обозначение --- $\langle\sigma\rangle$. Количество элементов в этой группе $\abs{\langle\sigma\rangle}$, равняется порядку $\sigma$ (почему?).
    \example $\langle\cycle{1,2}\rangle = \{e, \cycle{1,2}\}$.
    \problem  Положим $n = 4$. Опишите $\langle\sigma\rangle$ для \\
    \sub $\sigma = \cycle{1,2,3}$; \sub $\sigma = \cycle{3,2,4,1}$; \sub $\sigma = \cycle{1,2}\cycle{3,4}$.
\begin{solution}
    \sub $\langle\sigma\rangle = \{e, \cycle{1,2,3}, \cycle{1,3,2}\}$; \\
    \sub $\langle\sigma\rangle = \{e, \cycle{3,2,4,1}, \cycle{3,4}\cycle{2,1}, \cycle{3,1,4,2}\}$; \\
    \sub $\langle\sigma\rangle = \{e, \cycle{1,2}\cycle{3,4}\}$.
\end{solution}

    \defn \emph{Транспозициями} называются циклические перестановки порядка~2, то есть перестановки вида $\cycle{a,b}$.
    \problem Запишите все элементы $\sigma \in S_3$ в виде произведения транспозиций.
    \begin{solution}
        $e, \cycle{1,2}, \cycle{1,3}, \cycle{2,3}, \cycle{1,2}\cycle{2,3}, \cycle{2,3}\cycle{1,2}.$
    \end{solution}
    \par Доказательство следующей теоремы может вам напомнить решение задачки про последовательности нулей и единиц из листка 01 (Индукция).
    \begin{theorem} \label{produt_of_transpositions}
        Для любого $n \in \N$ любая перестановка $\sigma \in S_n$ может быть записана в виде произведения транспозиций.
    \end{theorem}
    \begin{proof}
        Докажем по индукции. Предположим, что для $n - 1$ это утверждение уже доказано. Обозначим за $x$ число, в которое $\sigma$ переводит $n$; формулой~--- $x = \sigma n$. Рассмотрим перестановку $\tau = \cycle{x,n} \circ \sigma$. Заметим, что $\tau$ оставляет число $n$ на месте, то есть мы можем считать, что $\tau \in S_{n-1}$. По предположению индукции, $\tau$ может быть записана в виде произведения транспозиций из $S_{n-1}$. Следовательно, $\sigma$ тоже может быть записана в виде произведения транспозиций (только уже из $S_n$). Если $\tau$ записывалась как $\tau = t_1 \circ t_2 \circ \cdots \circ t_k$, где все $t_i$ --- транспозиции, то $\sigma$ можно записать как $\sigma = \cycle{x,n} \circ t_1 \circ t_2 \circ \cdots \circ t_k$. 
    \end{proof}
    
    \defn Зафиксируем перестановку $\sigma \in S_n$. Пара чисел $1 \leq a < b \leq n$ назывется \emph{беспорядком} или \emph{инверсией} для $\sigma$, если $\sigma a > \sigma b$.
    \example Пусть $\sigma = \cycle{1,3}$. Тогда бепорядками будут пары $(1,2)$, $(1,3)$ и $(2,3)$.
    \problem Найдите беспорядки для всех перестановок $\sigma \in S_3$.
    \begin{solution}
        Беспорядки $e$ --- $\{\}$ (пустое множество);
        беспорядки $\cycle{1,2}$ --- $\{(1,2)\}$;
        беспорядки $\cycle{2,3}$ --- $\{(2,3)\}$;
        беспорядки $\cycle{1,3}$ --- $\{(1,2), (1,3), (2,3)\}$;
        беспорядки $\cycle{1,2,3}$ --- $\{(1,3), (2,3)\}$;
        беспорядки $\cycle{1,3,2}$ --- $\{(1,2), (1,3)\}$.
    \end{solution}
    
    \defn Перестановка называется \emph{чётной}, если у неё чётное число беспорядков, и \emph{нечётной} иначе.
    \example $e$, $\cycle{1,2,3}$ и $\cycle{1,3,2}$ --- чётные перестановки, $\cycle{1,2}$, $\cycle{2,3}$ и $\cycle{1,3}$ --- нечётные.
    
    \section{Задачи для самостоятельного решения}
    
    \problem Опишите $\langle\sigma\rangle$ для \\
    \sub $\sigma = \cycle{1,2,3}\cycle{4,5}$; \sub $\sigma = \cycle{1,2,3,4,5,6}\cycle{7,8,9}$.
    
    \problem Запишите в виде произведения транспозиций следующие перестановки: \\
    \sub $\cycle{1,2,3,4}$; \\ 
    \sub $\cycle{1,2,3,4}\cycle{5,6,7}$; \\ 
    \sub $\cycle{a,b,c,d,e}$; \\ 
    \sub $\cycle{1,2,\ldots,k-1,k}$.
    \defn Назовём \emph{элементарной транспозицией} транспозицию соседних чисел $\cycle{k,k+1}$.
    \problem Запишите в виде произведения элементарных транспозиций следующие перестановки: \\
    \sub $\cycle{1,3}$; \\ 
    \sub $\cycle{1,2,3}$; \\ 
    \sub $\cycle{1,3,2}$; \\ 
    \sub $\cycle{1,3,5}$; \\ 
    \sub $\cycle{1,3,5}\cycle{2,4}$; \\ 
    \sub $\cycle{1,k}$; \\ 
    \sub $\cycle{1,2,\ldots,k-1,k}$.
    \Problem{3} Докажите усиленную версию теоремы \ref{produt_of_transpositions}: любая перестановка может быть записана в виде произведения элементарных транспозиций.
    
    \problem Найдите беспорядки для следующих перестановок: \\
    \sub $\cycle{1,4}$; 
    \sub $\cycle{1,2,4}$; 
    \sub $\cycle{1,3,4}$; 
    \sub $\cycle{1,5}$; 
    \sub $\cycle{1,k}$. 
    
    \problem Докажите, что все транспозиции --- нечётные перестановки.
    \Problem{4} Докажите, что при умножении переcтановок их чётности <<складываются>>. \\ 
    Произведение двух чётных перестановок --- чётная перестановка. \\
    Произведение двух нечётных перестановок --- чётная перестановка. \\
    Произведение чётной и нечётной перестановок --- нечётная перестановка.
    \problem Докажите, что все циклы чётной длины --- нечётные перестановки, а все циклы нечётной длины~--- чётные перестановки.
\end{document} 