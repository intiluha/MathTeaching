\documentclass[a4paper,12pt]{article}

\usepackage{school}

\begin{document}
    \head{28 ноября}{Устная командная олимпиада}
    \textbf{Правила.} Обязательное требование --- члены команды должны сдавать задачи \textbf{по очереди}, но порядок самих задач может быть любым. Например (если в команде 3 человека), Вася сдал первую, потом Петя седьмую, потом Маша третью, потом снова Вася десятую и так далее. Длительность --- 100 минут. Удачи!
    \vspace{5pt}
    \hrule
    \vspace{5pt}
    
    % ------------------------------1 сложность------------------------------
    \problem Может ли число $2^{44} + 2^{99} + 2^n$ быть полным квадратом?
\begin{solution}
    Положим $n = 2 \cdot 76$. Тогда
    \begin{equation*}
        (2^{22} + 2^{76})^2 = (2^{22})^2 + 2 \cdot 2^{22} \cdot 2^{76} + (2^{76})^2 = 2^{44} + 2^{99} + 2^n.
    \end{equation*}
\end{solution}
    \problem Может ли натуральное число --- палиндром из 666 цифр быть простым?
    \begin{solution}
        Нет, не может, так как оно делится на 11. В самом деле, признак делимости на 11  говорит, что число делится на 11, тогда и только тогда, когда его знакопеременная сумма цифр делится на 11 (точнее, сумма цифр на нечётных позициях, минус сумма цифр на чётных позициях). Но для палиндрома чётной длины эта сумма всегда равна нулю. Действительно, если на $k$-ом месте стоит цифра $x$, то на $666 - k + 1$ месте стоит та же цифра $x$, а чётности $k$ и $666 - k + 1$ всегда разные.
    \end{solution}
    
    % ------------------------------2 сложность------------------------------
    \problem Найдите все натуральные $m$ и $n$, такие что $m! + 12 = n^2$.
    \begin{solution}
        Предположим, что $m \geq 6$. Тогда $m!$ делится на 9, а значит $m! + 12$ делится на 3, но не делится на 9. Так как квадрат не может делиться на 3, но не делиться на 9, получаем противоречие. Следовательно, $m < 6$. Перебором устанавливаем, что из меньших чисел подходит только $m = 4$ и $n = 6$.
    \end{solution}
    \problem В турнире каждый сыграл с каждым ровно один раз, причём за победу, ничью и поражение давалось 2, 1 и 0 очков соответственно. Выяснилось, что все участники набрали одинаковое число очков, причём если удалить любого участника и аннулировать результаты всех игр с ним, то число очков у всех остальных участников останется одинаковым. Докажите, что все партии кончились вничью.
    \begin{solution}
        Предположим, хотя бы одна игра окончилась не вничью. Скажем, Вася обыграл Петю. Если бы Петя проиграл все игры, то у него не могло бы быть столько же очков, сколько и у Васи. Скажем, Петя не проиграл (то есть выиграл или сыграл вничью) Маше. Тогда, если удалить из турнира Петю, то количество очков Васи уменьшится на 2, а количество очков Маши уменьшится на 1 или 0. В любом случае, очков у них станет не поровну, что противоречит условию.
    \end{solution}
    
    % ------------------------------3 сложность------------------------------
    \problem Рассмотрим все моменты времени, когда часовая и минутная стрелки часов лежат на одной прямой. Найдутся ли среди таких прямых две взаимно перпендикулярные?
    \begin{solution}
        Пусть такие два момента времени найдутся. Тогда между ними часовая стрелка должна была повернуться на угол, кратный $90^{\circ}$, а значит прошло время, кратное трём часам. Но тогда минутная стрелка должна сделать целое число оборотов и остаться на том же месте. Противоречие.
    \end{solution}
    \problem В вершинах куба расставлены числа от 1 до 8. На каждой грани записали сумму чисел, расставленных в её вершинах. Могли ли получиться шесть последовательных натуральных чисел?
    \begin{solution}
        Каждая вершина лежит ровно на трёх гранях куба. Следовательно, сумма чисел на гранях равняется утроенной сумме чисел в вершинах куба, то есть $3 \cdot \frac{8 \cdot 9}{2} = 108$. Предположим, на гранях получились шесть последовательных чисел, от $n$ до $n + 5$. Тогда сумма этих чисел равнялась бы $6n + 15$. Следовательно, имеем $108 = 6n + 15$. Однако это уравнение не имеет целых решений (108 делится на 6, а 15 не делится), противоречие.
    \end{solution}
    
    % ------------------------------4 сложность------------------------------
    \problem Решите уравнение:
    \begin{equation*}
        (1 + x + x^2 + \ldots + x^7) (1 + x + x^2 + \ldots + x^5) = (1 + x + x^2 + \ldots + x^6)^2.
    \end{equation*}
    \begin{solution}
        \begin{lemma}
            Для любого $k$ выполнено равенство
            \begin{equation*}
                (1 - x)(1 + x + \cdots + x^k) = 1 - x^{k + 1}
            \end{equation*}
        \end{lemma}
        \begin{proof}
            Для $k = 1$ утверждение легко проверить. Покажем по индукции, что оно верно для всех $k$.
            \begin{equation*}
                (1 - x)(1 + x + \cdots + x^k) = (1 - x)(1 + x + \cdots + x^{k-1}) + (1 - x)x^k = 1 - x^k + x^k - x^{k+1} = 1 - x^{k+1}
            \end{equation*}
        \end{proof}
        Умножим обе части на $(1 - x)^2$. Ввиду леммы, выражение примет вид
        \begin{equation*}
            (1 - x^8)(1 - x^6) = (1 - x^7)^2.
        \end{equation*}
        Раскрыв скобки, получим
        \begin{equation*}
            1 - x^6 - x^8 + x^{14} = 1 - 2x^7 + x^{14}.
        \end{equation*}
        Перенеся всё в одну сторону и приведя подобные, получим
        \begin{equation*}
            x^6(1 - x)^2 = 0.
        \end{equation*}
        Таким образом, исходное уравнение (до умножения на $(1 - x)^2$) эквивалентно $x^6 = 0$.
    \end{solution}
    \problem На плоскости отмечено несколько точек и рассматриваются всевозможные отрезки \textbf{с концами в отмеченных точках}. Назовём отрезок \emph{чётным}, если на нём чётное число отмеченных точек, и \emph{нечётным} иначе. Докажите, что чётных отрезков больше.
    \begin{solution}
        Сперва докажем это утверждение для прямой вместо плоскости. Если отмечено лишь 2 точки, то есть один чётный отрезок и ноль нечётных. Если мы отмечаем новую точку, находящуюся по одну сторону от всех уже выставленных, то добавляется либо поровну чётных и нечётных отрезков (если было чётное число точек), либо больше чётных, чем нечётных (если было нечётное число точек). В любом случае, после добавления любого числа точек чётных отрезков будет всё ещё строго больше, чем нечётных. \par
        Теперь докажем это утверждение для плоскости. Выберем все прямые в плоскости, на которые попадает хотя бы две отмеченные точки. На каждой из таких прямых, как мы уже доказали, чётных отрезков больше, чем нечётных. Но тогда и на всей плоскости чётных отрезков больше, чем нечётных.
    \end{solution}
    \problem  В некоторых клетках таблицы $1501 \times 1501$ стоят крестики. Каждый крестик является единственным либо в своей строке, либо в своём столбце. Какое наибольшее число крестиков может стоять в таблице?
    \begin{solution}
        3000 крестиков можно поставить, например, таким образом: заполнить всю первую строку и весь первый столбец, а после стереть угловой крестик. Легко видеть, что условия выполнены. \par
        Будем доказывать, что на поле $n \times n$ можно поставить не более $2n - 2$ крестиков. Для поля $2 \times 2$ (то есть для $n = 2$) это очевидно. Индукция по $n$. Предположим, что мы поставили больше $2n$ крестиков на поле $(n+1) \times (n+1)$. Ясно, что какой-то из них (назовём его $A$) должен быть единственным в своей строке и какой-то из них (назовём его $B$) должен быть единственным в своём столбце. Тогда выбросим из нашей таблицы строку, содержащую $A$ и столбец, содержащий $B$. Получим таблицу $n \times n$, в которой стоят больше $2n - 2$ крестиков, что противоречит предположению индукции.
    \end{solution}
    
    % ------------------------------? сложность------------------------------
    \problem Сколько есть перестановок $\sigma \in S_n$, таких что $\sigma^2 = e$?
    \begin{solution}
        Все такие перестановки записываются как произведение нескольких непересекающихся транспозиций. Транспозиций всего: $\frac{n(n-1)}{2}$; произведений двух независимых транспозиций всего: $\dfrac{\frac{n(n-1)}{2}\frac{(n-2)(n-3)}{2}}{2!}$; произведений $k$ независимых транспозиций аналогично будет
        \begin{equation*}
            \frac{\frac{n(n-1)}{2}\frac{(n-2)(n-3)}{2}\ldots\frac{(n-2k+2)(n-2k+1)}{2}}{k!},
        \end{equation*}
        что может быть записано как
        \begin{equation*}
            \frac{n!}{2^k(n-2k)!k!}.
        \end{equation*}
        Следовательно, ответ на задачу:
        \begin{equation*}
            \frac{n!}{2^1(n-2)!1!} + \frac{n!}{2^2(n-4)!2!} + \cdots + \frac{n!}{2^k(n-2k)!k!} + \cdots + \frac{n!}{2^{\left[\frac{n}{2}\right]}(n-2\left[\frac{n}{2}\right])!\left[\frac{n}{2}\right]!}.
        \end{equation*}
    \end{solution}
\end{document}