\documentclass[a4paper,12pt]{article}

\usepackage{school}

\begin{document}
    \head{26 октября}{Группы симметрий геометрических фигур}
    
    \section{Разбор}
    
    \textbf{Обозначения.} Множество целых чисел обозначается символом $\Z$, натуральных --- $\N$, рациональных --- $\Q$, вещественных --- $\R$. Как учил нас Декарт, точки плоскости можно отождествить с их координатами, поэтому множество точек плоскости будем обозначать $\R \times \R$ или просто $\R^2$. \\
    Значок $\subset$ обозначает <<является подмножеством>>. Например: $\{1,2\}~\subset~\{1,2,3\}$, \quad $\N~\subset~\Z$,\quad $\Z~\subset~\Q$,\quad $\Q~\subset~\R$,\quad $\{1, 2, 3\}~\subset~\{1, 2, 3\}$.
    
    \defn \emph{Симметриями} фигуры $\Phi~\subset~\R^2$ называются такие движения плоскости $A \colon \R^2 \to \R^2$, которые оставляют эту фигуру на месте, то есть
    \begin{equation*}
        \forall x (x \in \Phi \Rightarrow A(x) \in \Phi).
    \end{equation*}
    Множество всех симметрий фигуры $\Phi$ будем обозначать $\Sym(\Phi)$.
    
    \begin{wrapfigure}{r}{0.4\linewidth}
        \vspace{-1ex}
        \begin{tikzpicture}[scale=3]
            \draw[thick] (0,0) node[left]{$A$} -- (0:2) node[right]{$B$}  -- (60:2) node[above]{$C$} -- cycle;
            \draw[ultra thin, green!50!black, name path=a] (0,0)+(30:-0.3) -- +(30:2);
            \draw[ultra thin, green!50!black, name path=b] (0:2)+(150:-0.3) -- +(150:2);
            \draw[ultra thin, green!50!black] (60:2)+(270:-0.3) -- +(270:2);
            \draw[name intersections={of=a and b, by=x}] (x) node[right]{$O$};
        \end{tikzpicture}
    \end{wrapfigure}
    \example Пусть $P_3$ --- правильный треугольник, смотри рисунок. Тогда множество $\Sym(P_3)$ состоит из:\\
    $e$ --- тождественного движения;\\
    $r_{120}$ --- поворота относительно $O$ на $120^\circ$ (против часовой стрелки);\\
    $r_{240}$ --- поворота относительно $O$ на $240^\circ$ (против часовой стрелки);\\
    $s_C$ --- симметрии относительно серединного перпендикуляра к $AB$;\\
    $s_A$ --- симметрии относительно серединного перпендикуляра к $BC$;\\
    $s_B$ --- симметрии относительно серединного перпендикуляра к $CA$.
    
    Очевидно, что композиция двух симметрий какой-либо фигуры также является симметрией той же фигуры. Например, $s_C \circ s_B = r_{240}$.
    
    \problem Заполните <<таблицу умножения>> для множества $\Sym(P_3)$ с операцией $\circ$.
    \begin{center}
        \begin{equation*}
        \newcommand{\tab}{\phantom{r_{240}}}
            \begin{array}{c||c|c|c|c|c|c}
                \circ   & e       & r_{120} & r_{240} & s_C     & s_A     & s_B     \\ \hline \hline
                e       & e       & r_{120} & r_{240} & s_C     & s_A     & s_B     \\ \hline
                r_{120} & r_{120} & r_{240} & e       & s_B     & s_C     & s_A     \\ \hline
                r_{240} & r_{240} & e       & r_{120} & s_A     & s_B     & s_C     \\ \hline
                s_C     & s_C     & s_A     & s_B     & e       & r_{120} & r_{240} \\ \hline
                s_A     & s_A     & s_B     & s_C     & r_{240} & e       & r_{120} \\ \hline
                s_B     & s_B     & s_C     & s_A     & r_{120} & r_{240} & e
            \end{array}
        \end{equation*}
    \end{center}
    
    Множество с бинарной (то есть принимающей два аргумента) операцией называется \emph{группой}, если эта операция удовлетворяет некоторым требованиям. Мы не будем сейчас перечислять эти требования, но постулируем, что для любой фигуры $\Phi$, множество симметрий $\Sym(\Phi)$ с операцией композиции $\circ$ является группой.
    
    \section{Задачи для самостоятельного решения}
    
    \problem Пусть $X$ --- множество параллелограммов, $Y$ --- множество ромбов, $Z$ --- множество прямоугольников, $W$ --- множество четырёхугольников с перпендикулярными диагоналями. \\
    \sub Верны ли следующие утверждения? $Y \subset X$;  $Z \subset W$; $Y = X \cap W$; $Z \subset X$. \\
    \sub Выразите множество квадратов $Q$ через эти множества, используя операции объединения и/или пересечения. 
    
    \problem Опишите множества симметрий следующих фигур: \\
    \sub $\Phi_1 \in (X - Z) - W$ --- параллелограмм общего вида; \\
    \sub $\Phi_2 \in Z - W$ --- прямоугольник общего вида; \\
    \sub $\Phi_3 \in Y - Z$ --- ромб общего вида; \\
    \sub $P_4$ --- квадрат (он же правильный четырёхугольник); \\
    \sub $P_5$ --- правильный пятиугольник; \\
    \Sub{2} $P_{2n}$ --- правильный ${2n}$-угольник; \\
    \Sub{2} $P_{2n + 1}$ --- правильный ${2n + 1}$-угольник; \\
    \Sub{2} $D$ --- круг.
    
    \problem Постройте таблицы умножения для: \\
    \sub $(\Sym(\Phi_1), \circ)$; \\
    \sub $(\Sym(\Phi_2, \circ)$; \\
    \sub $(\Sym(\Phi_3, \circ)$; \\
    \sub $(\Sym(P_4), \circ)$; \\
    \Sub{2} $(\Sym(P_5), \circ)$; \\
    \Sub{3} $(\Sym(P_{2n}), \circ)$; \\
    \Sub{3} $(\Sym(P_{2n + 1}), \circ)$; \\
    \Sub{3} $(\Sym(D), \circ)$.
\end{document} 