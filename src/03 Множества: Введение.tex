\documentclass[a4paper,12pt]{article}

\usepackage{school}

\begin{document}
    \head{12 октября}{Множества: Введение}
    \rules{18 октября}{10}
    
    \section{Задачи на разбор}
    
    \textbf{Определение.} Множеством называется набор из любого числа объектов, отобранных по какому-нибудь принципу, но без повторений. Принцип может быть сколь угодно хитрым, как вы увидите в примерах. Множества обозначают фигурными скобками, внутрь которых включают либо полный список элементов, либо правило, которое их выделяет.
    
    \textbf{Примеры.}
    \newcommand{\tab}{\rule{2mm}{0pt}}
    \begin{eqnarray*}
        A &=& \{ \text{ученики класса 8--4} \} \\
        B &=& \{ \text{люди в этом дискорд-канале} \} \\
        C &=& \{ \text{стулья в аудитории 404} \} \\
        D &=& \{ \text{натуральные числа} \} \\
        E &=& \{ \text{точки плоскости} \} \\
        F &=& \{ 0, \tab 7, \tab \text{Илон Маск, \tab левый стул первой парты среднего ряда}, \tab (\sqrt{2}, \sqrt{3}) \}
    \end{eqnarray*}
    Обратите внимание, что определение не требует, чтобы объекты из одного множества были в каком-нибудь смысле <<одной природы>>, последний пример это иллюстрирует. Однако, на практике чаще всего полезны именно однородные множества.
    
    \problem Сколько элементов в каждом из множеств $A, \ldots, F$?
\begin{solution}
    $\abs{A} = 25$, $\abs{B} = 11$, $\abs{C} \approx 30$, $\abs{D} = \infty$ (точнее --- $\aleph_0$, но это пока не нужно понимать), $\abs{E} = \infty$ (точнее --- $2^{\aleph_0}$, но это пока не нужно понимать), $\abs{F} = 5$.
\end{solution}
    
    \textbf{Определение.} Здесь и далее незанятыми заглавными латинскими буквами обозначаются произвольные множества. Объединением множеств $X$ и $Y$ называется множество $X \cup Y$, содержащее в себе в точности те элементы, которые есть в $X$ \textbf{или} в $Y$ (обратите внимание, что <<или>> не исключающее). Пересечением множеств $X$ и $Y$ называется множество $X \cap Y$, содержащее в себе в точности те элементы, которые есть \textbf{и} в $X$, \textbf{и} в $Y$.
    
    \problem Сколько элементов в множествах: \sub $A \cup B$; \sub $A \cap B$; \sub \label{CDEF} $((C \cup D) \cup E) \cap F$?
    \begin{solution}
        \sub $\abs{A \cup B} = \abs{\{\text{ученики 8-4 и Илья Левин}\}} = 26$; \\
        \sub $\abs{A \cap B} = \abs{\{\text{ученики 8-4 в этом дискорд-канале}\}} = 10$; \\
        \sub $\abs{((C \cup D) \cup E) \cap F} = \abs{\{ 7, \tab \text{левый стул первой парты среднего ряда}, \tab (\sqrt{2}, \sqrt{3}) \}} = 3$.
    \end{solution}
    
    \textbf{Обозначения.} $\forall$ обозначает <<для любого>>; $\exists$ обозначает <<существует>>; \\
    $x \in S$ обозначает <<элемент $x$ лежит в множестве $S$>>; \\ 
    $x \notin S$ обозначает <<элемент $x$ \textbf{не} лежит в множестве $S$>>; \\ 
    $P \Rightarrow Q$ обозначает <<если $P$, то $Q$>> (или, другими словами, <<из $P$ следует $Q$>>).
    
    \problem Верны ли следующие утверждения? \\
    \sub $\text{Михаил Лесс} \in A$; \\
    \sub $\text{Диана Варзина} \notin B$; \\
    \sub $\forall x (x \in B \Rightarrow x \in A)$; \\
    \sub $\exists x (x \in C \text{ и } x \in F)$.
    \begin{solution}
        \sub Верно. \sub Неверно. \\ 
        \sub Словами: любой человек в этом дискорд-канале является учеником 8-4; неверно. \\
        \sub Словами: существует стул из 404 аудитории, лежащий в множестве $F$; верно. 
    \end{solution}
    
    \problem Пусть $X$ и $Y$ --- какие-то множества. Всегда ли верны следующие утверждения? \\
    \sub $\forall x (x \in X \Rightarrow x \in X \cup Y)$; \\
    \sub $\forall x (x \in X \Rightarrow x \in X \cap Y)$; \\
    \sub $\forall x (x \in X \text{ и } x \in Y \Rightarrow x \in X \cap Y)$.
    \begin{solution}
        \sub Словами: всякий элемент множества $X$ лежит в объединении $X$ и $Y$. Верно, смотри определение. \\
        \sub Словами: всякий элемент множества $X$ лежит в пересечении $X$ и $Y$. Неверно, контрпример: $X = \{1, 2\}, \quad Y = \{3\}, \quad x = 1$. Тогда $x$ лежит в $X$, но не лежит в $X \cap Y = \emptyset$.  \\
        \sub Словами: всякий элемент, лежащий и в $X$, и в $Y$, лежит в их пересечении. Верно, смотри определение.
    \end{solution}
    
    \section{Задачи для самостоятельного решения}
    
    \problem Придумайте множество из \sub 6; \sub 100; \sub 0 элементов.
    
    \problem Придумайте такие множества $G$ и $H$, что в них по 5 элементов, а в их пересечении 2 элемента. Сколько элементов в их объединении? Всегда ли так будет?
    
    \Problem{2} Придумайте множество $I$, удовлетворяющее условию
    \begin{equation*}
        \forall x (x \in D \Rightarrow \exists y (y \in I \text{ и } y > x)).
    \end{equation*}
    
    \problem Пусть $X$ и $Y$ --- какие-то множества. Всегда ли верны следующие утверждения? \\
    \sub $\forall x (x \in X \cup Y \Rightarrow x \in X)$; \\
    \sub $\forall x (x \in X \cap Y \Rightarrow x \in X)$; \\
    \sub $\forall x (x \in X \cup Y \Rightarrow x \in X \text{ или } x \in Y)$.
    
    \problem Зависит ли ответ на задачу \ref{CDEF} из разбора от расстановки скобок?
    
    \problem Сформулируйте и докажите. \\
    \sub ассоциативность объединения; \\
    \sub ассоциативность пересечения; \\
    \sub коммутативность объединения; \\
    \sub коммутативность пересечения; \\
    \Sub{по 2} дистрибутивность (да-да, их и вправду две разных!) \\
    \textit{Подсказка: все эти утверждения очевидны из определения, надо только записать их значками.}
\end{document} 